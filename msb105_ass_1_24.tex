% Options for packages loaded elsewhere
\PassOptionsToPackage{unicode}{hyperref}
\PassOptionsToPackage{hyphens}{url}
\PassOptionsToPackage{dvipsnames,svgnames,x11names}{xcolor}
%
\documentclass[
  a4paper,
]{article}

\usepackage{amsmath,amssymb}
\usepackage{iftex}
\ifPDFTeX
  \usepackage[T1]{fontenc}
  \usepackage[utf8]{inputenc}
  \usepackage{textcomp} % provide euro and other symbols
\else % if luatex or xetex
  \usepackage{unicode-math}
  \defaultfontfeatures{Scale=MatchLowercase}
  \defaultfontfeatures[\rmfamily]{Ligatures=TeX,Scale=1}
\fi
\usepackage{lmodern}
\ifPDFTeX\else  
    % xetex/luatex font selection
\fi
% Use upquote if available, for straight quotes in verbatim environments
\IfFileExists{upquote.sty}{\usepackage{upquote}}{}
\IfFileExists{microtype.sty}{% use microtype if available
  \usepackage[]{microtype}
  \UseMicrotypeSet[protrusion]{basicmath} % disable protrusion for tt fonts
}{}
\makeatletter
\@ifundefined{KOMAClassName}{% if non-KOMA class
  \IfFileExists{parskip.sty}{%
    \usepackage{parskip}
  }{% else
    \setlength{\parindent}{0pt}
    \setlength{\parskip}{6pt plus 2pt minus 1pt}}
}{% if KOMA class
  \KOMAoptions{parskip=half}}
\makeatother
\usepackage{xcolor}
\setlength{\emergencystretch}{3em} % prevent overfull lines
\setcounter{secnumdepth}{5}
% Make \paragraph and \subparagraph free-standing
\ifx\paragraph\undefined\else
  \let\oldparagraph\paragraph
  \renewcommand{\paragraph}[1]{\oldparagraph{#1}\mbox{}}
\fi
\ifx\subparagraph\undefined\else
  \let\oldsubparagraph\subparagraph
  \renewcommand{\subparagraph}[1]{\oldsubparagraph{#1}\mbox{}}
\fi


\providecommand{\tightlist}{%
  \setlength{\itemsep}{0pt}\setlength{\parskip}{0pt}}\usepackage{longtable,booktabs,array}
\usepackage{calc} % for calculating minipage widths
% Correct order of tables after \paragraph or \subparagraph
\usepackage{etoolbox}
\makeatletter
\patchcmd\longtable{\par}{\if@noskipsec\mbox{}\fi\par}{}{}
\makeatother
% Allow footnotes in longtable head/foot
\IfFileExists{footnotehyper.sty}{\usepackage{footnotehyper}}{\usepackage{footnote}}
\makesavenoteenv{longtable}
\usepackage{graphicx}
\makeatletter
\def\maxwidth{\ifdim\Gin@nat@width>\linewidth\linewidth\else\Gin@nat@width\fi}
\def\maxheight{\ifdim\Gin@nat@height>\textheight\textheight\else\Gin@nat@height\fi}
\makeatother
% Scale images if necessary, so that they will not overflow the page
% margins by default, and it is still possible to overwrite the defaults
% using explicit options in \includegraphics[width, height, ...]{}
\setkeys{Gin}{width=\maxwidth,height=\maxheight,keepaspectratio}
% Set default figure placement to htbp
\makeatletter
\def\fps@figure{htbp}
\makeatother
% definitions for citeproc citations
\NewDocumentCommand\citeproctext{}{}
\NewDocumentCommand\citeproc{mm}{%
  \begingroup\def\citeproctext{#2}\cite{#1}\endgroup}
\makeatletter
 % allow citations to break across lines
 \let\@cite@ofmt\@firstofone
 % avoid brackets around text for \cite:
 \def\@biblabel#1{}
 \def\@cite#1#2{{#1\if@tempswa , #2\fi}}
\makeatother
\newlength{\cslhangindent}
\setlength{\cslhangindent}{1.5em}
\newlength{\csllabelwidth}
\setlength{\csllabelwidth}{3em}
\newenvironment{CSLReferences}[2] % #1 hanging-indent, #2 entry-spacing
 {\begin{list}{}{%
  \setlength{\itemindent}{0pt}
  \setlength{\leftmargin}{0pt}
  \setlength{\parsep}{0pt}
  % turn on hanging indent if param 1 is 1
  \ifodd #1
   \setlength{\leftmargin}{\cslhangindent}
   \setlength{\itemindent}{-1\cslhangindent}
  \fi
  % set entry spacing
  \setlength{\itemsep}{#2\baselineskip}}}
 {\end{list}}
\usepackage{calc}
\newcommand{\CSLBlock}[1]{\hfill\break\parbox[t]{\linewidth}{\strut\ignorespaces#1\strut}}
\newcommand{\CSLLeftMargin}[1]{\parbox[t]{\csllabelwidth}{\strut#1\strut}}
\newcommand{\CSLRightInline}[1]{\parbox[t]{\linewidth - \csllabelwidth}{\strut#1\strut}}
\newcommand{\CSLIndent}[1]{\hspace{\cslhangindent}#1}

\makeatletter
\@ifpackageloaded{caption}{}{\usepackage{caption}}
\AtBeginDocument{%
\ifdefined\contentsname
  \renewcommand*\contentsname{Table of contents}
\else
  \newcommand\contentsname{Table of contents}
\fi
\ifdefined\listfigurename
  \renewcommand*\listfigurename{List of Figures}
\else
  \newcommand\listfigurename{List of Figures}
\fi
\ifdefined\listtablename
  \renewcommand*\listtablename{List of Tables}
\else
  \newcommand\listtablename{List of Tables}
\fi
\ifdefined\figurename
  \renewcommand*\figurename{Figure}
\else
  \newcommand\figurename{Figure}
\fi
\ifdefined\tablename
  \renewcommand*\tablename{Table}
\else
  \newcommand\tablename{Table}
\fi
}
\@ifpackageloaded{float}{}{\usepackage{float}}
\floatstyle{ruled}
\@ifundefined{c@chapter}{\newfloat{codelisting}{h}{lop}}{\newfloat{codelisting}{h}{lop}[chapter]}
\floatname{codelisting}{Listing}
\newcommand*\listoflistings{\listof{codelisting}{List of Listings}}
\makeatother
\makeatletter
\makeatother
\makeatletter
\@ifpackageloaded{caption}{}{\usepackage{caption}}
\@ifpackageloaded{subcaption}{}{\usepackage{subcaption}}
\makeatother
\ifLuaTeX
\usepackage[bidi=basic]{babel}
\else
\usepackage[bidi=default]{babel}
\fi
\babelprovide[main,import]{american}
% get rid of language-specific shorthands (see #6817):
\let\LanguageShortHands\languageshorthands
\def\languageshorthands#1{}
\ifLuaTeX
  \usepackage{selnolig}  % disable illegal ligatures
\fi
\usepackage{bookmark}

\IfFileExists{xurl.sty}{\usepackage{xurl}}{} % add URL line breaks if available
\urlstyle{same} % disable monospaced font for URLs
\hypersetup{
  pdftitle={How can Quarto documents help with reproducibility in research?},
  pdfauthor={Håkon Ivesdal},
  pdflang={en-US},
  colorlinks=true,
  linkcolor={blue},
  filecolor={Maroon},
  citecolor={Blue},
  urlcolor={Blue},
  pdfcreator={LaTeX via pandoc}}

\title{How can \emph{Quarto} documents help with reproducibility in
research?}
\author{Håkon Ivesdal}
\date{Friday 8 Nov, 2024}

\begin{document}
\maketitle
\begin{abstract}
There is a recent reproducibility crisis in research. \emph{Quarto}
documents can help by enhancing reproducibility in research by
integrating code and data directly in the document. The Quarto document
could then be the entire submission to the journals.
\end{abstract}

\section{Introduction}\label{introduction}

Scientific progress relies on the generation of robust, reliable
knowledge that provides a solid foundation for future advancements.
There is strong evidence suggesting that many of these findings may not
endure over time. This reproducibility crisis in basic and preclinical
research is largely attributed to a lack of adherence to proper
scientific practices, compounded by the pressure to ``publish or
perish'' (Begley \& Ioannidis, 2015). Reproducing an experiment is a key
method scientists use to strengthen confidence in their conclusions and
data handling is an important part of it (McNutt, 2014).

Unfortunately the long-established archives at economics journals do not
effectively support the reproduction of published results. Data-only
archives, like those of the Journal of Business and Economic Statistics
and the Economic Journal, are limited in part because many authors do
not provide their data. Similarly, results published in the FRB
St.~Louis Review are rarely reproducible, even when data and code are
included in the journal's archive (McCullough et al., 2008).

\section{Terminology}\label{terminology}

Reproduce and replicate have been used differently and indifferently
from one and another, depending on the situation and the field. Normally
to suggest a form of duplication of a study. In computational fields,
reproducibility typically refers to the ability to reproduce
computational results, focusing specifically on sharing and properly
documenting data and code (Peng, 2011).

When people refer to a ``replication'', they may mean one of the
following: duplicating the methods, procedures, and analysis of a study
(regardless of the outcomes) or duplicating the methods, procedures,
analysis, and the results themselves (Fidler \& Wilcox, 2021). It is
implied that the new data set determine the outcome as the other
variables are constant. These meanings describes two of the three parts
that make up Research reproducibility. The former describes Methods
reproducibility, while the latter describes Results reproducibility.
Robustness and generalizability is the third part and takes into account
the difference that outside effects have on the experimental framework's
setting (Goodman et al., 2016).

The NSF (Subcommittee on Replicability in Science Advisory Committee to
the National Science Foundation Directorate for Social, Behavioural, and
Economic Sciences) defines Methods reproducibility simply as
Reproducibility and Results reproducibility as Replicability (Bollen et
al., 2015). This is also how it will be referred to in the rest of the
paper.

\section{How can Quarto documents help with
reproducibility?}\label{how-can-quarto-documents-help-with-reproducibility}

Reproducibility is a precursor to achieving replicability, but it can be
challenging to attain. Reproducibility demands that the code and data
used to generate a manuscript is fully integrated into it (\emph{About
Quarto}, n.d.). In traditional documents third party software are used
to make the different models and calculations. \emph{R Markdown}, which
\emph{Quarto} is based on, enables the creation of computationally
reproducible documents by allowing authors to embed \emph{R} code for
data processing, analysis, exploration, table generation, and
visualization directly into structured electronic documents. These
documents consist of sections of \emph{R} code, referred to as
computational components because they are produced through computational
methods, along with narrative components. In scientific writing, these
narrative sections provide context, explain background, define
objectives, set themes, and communicate results. The documents can then
be rendered into formats such as .html, .pdf, .doc (Grolemund, n.d.).

Quarto is the next generation version of \emph{R Markdown}, without
relying or requiring \emph{R}. Designed to be a multilingual tool from
the start, by supporting languages such as \emph{R}, \emph{Python},
\emph{JavaScript}, and \emph{Julia}, with the goal of being adaptable to
future programming languages that may emerge in the future (\emph{FAQ
for R Markdown Users}, n.d.). A key part of the adaptability and
functionality of \emph{Quarto} documents are called packages and could
remove the need for third party applications. Packages are collections
of functions, data, and compiled code (\emph{R Packages}, n.d.). Some of
these extensions are default or downloadable in the integrated
development environment called \emph{RStudio} (\emph{R Studio -
Kunnskapsbasen - NTNU}, n.d.).

By integrating every part of the work after data extraction, the entire
entry to a Journal could just be one \emph{Quarto} document. This would
ease the process of achieving the same results and be a helpful step
towards replication.

\subsection{What problems remains and how can these be
solved?}\label{what-problems-remains-and-how-can-these-be-solved}

According to Wickham et al. (2019) Quarto documents are fully
reproducible, assuming hardware and software is accounted for. It is
therefore important to document hardware, system, system-libraries,
\emph{R}-version and package-versions. The first four are relative
easily done by using the built-in function: \texttt{sessionInfo()}.
However the packages are constantly updated and could be updated under
the development of the document. A solution to this problem is a package
in \emph{RStudio} called \emph{renv}. It enables the creation of
reproducible environments for \emph{R} projects, ensuring consistent
dependencies and package versions across different systems. Each project
is given its own private package library, meaning that installing or
updating a package for one project won't affect the other projects
(Ushey, n.d.).

\section{Conclusion}\label{conclusion}

The ability to reproduce results is foundational to scientific research
and replication. By embedding code and data directly within the
document, \emph{Quarto} can ensure that the methods, analyses and
calculation used to generate results are transparent and easily
accessible. While it cannot retrospectively fix past entries, it can
pave the way for more reliable published research findings in the
future.

\section*{References}\label{references}
\addcontentsline{toc}{section}{References}

\phantomsection\label{refs}
\begin{CSLReferences}{1}{0}
\bibitem[\citeproctext]{ref-zotero-97}
\emph{About Quarto}. (n.d.). \url{https://quarto.org/about.html}

\bibitem[\citeproctext]{ref-begley2015}
Begley, C. G., \& Ioannidis, J. P. A. (2015). Reproducibility in
science. \emph{Circulation Research}, \emph{116}(1), 116--126.
\url{https://doi.org/10.1161/CIRCRESAHA.114.303819}

\bibitem[\citeproctext]{ref-bollen2015}
Bollen, K., Cacioppo, J. T., Krosnick, J. A., Olds, J. L., \& Kaplan, R.
M. (2015). \emph{Social, {Behavioral}, and {Economic Sciences
Perspectives} on {Robust} and {Reliable Science}} (Report of the
Subcommittee on Replicability in Science Advisory Committee to the
National Science Foundation Directorate for Social, Behavioral, and
Economic Sciences). {NSF}.

\bibitem[\citeproctext]{ref-zotero-84}
\emph{FAQ for R Markdown Users}. (n.d.).
\url{https://quarto.org/docs/faq/rmarkdown.html}

\bibitem[\citeproctext]{ref-fidler2021}
Fidler, F., \& Wilcox, J. (2021). \emph{Reproducibility of scientific
results} (E. N. Zalta, Ed.; Summer 2021). Metaphysics Research Lab,
Stanford University.
\url{https://plato.stanford.edu/archives/sum2021/entries/scientific-reproducibility/}

\bibitem[\citeproctext]{ref-goodman2016}
Goodman, S. N., Fanelli, D., \& Ioannidis, J. P. A. (2016). What {Does
Research Reproducibility Mean}? \emph{Science Translational Medicine},
\emph{8}(341), 341ps12--341ps12.
\url{https://doi.org/10.1126/scitranslmed.aaf5027}

\bibitem[\citeproctext]{ref-grolemunda}
Grolemund, G., J. J. Allaire. (n.d.). \emph{R {Markdown}: {The
Definitive Guide}}. Retrieved September 15, 2021, from
\url{https://bookdown.org/yihui/rmarkdown/}

\bibitem[\citeproctext]{ref-mccullough2008}
McCullough, B. D., McGeary, K. A., \& Harrison, T. D. (2008). Do
economics journal archives promote replicable research? \emph{Canadian
Journal of Economics/Revue Canadienne d'économique}, \emph{41}(4),
1406--1420. \url{https://doi.org/10.1111/j.1540-5982.2008.00509.x}

\bibitem[\citeproctext]{ref-mcnutt2014}
McNutt, M. (2014). Reproducibility. \emph{Science}, \emph{343}(6168),
229--229. \url{https://doi.org/10.1126/science.1250475}

\bibitem[\citeproctext]{ref-peng2011}
Peng, R. D. (2011). Reproducible {Research} in {Computational Science}.
\emph{Science}, \emph{334}(6060), 1226--1227.
\url{https://doi.org/10.1126/science.1213847}

\bibitem[\citeproctext]{ref-zotero-101}
\emph{R Packages}. (n.d.). \url{https://www.datacamp.com/doc/r/packages}

\bibitem[\citeproctext]{ref-zotero-103}
\emph{R studio - kunnskapsbasen - NTNU}. (n.d.).
\url{https://i.ntnu.no/wiki/-/wiki/Norsk/R+Studio}

\bibitem[\citeproctext]{ref-ushey}
Ushey, K. (n.d.). \emph{renv/vignettes/renv.Rmd at main · rstudio/renv}.
\url{https://github.com/rstudio/renv/blob/main/vignettes/renv.Rmd}

\bibitem[\citeproctext]{ref-wickham2019}
Wickham, H., Averick, M., Bryan, J., Chang, W., McGowan, L., François,
R., Grolemund, G., Hayes, A., Henry, L., Hester, J., Kuhn, M., Pedersen,
T., Miller, E., Bache, S., Müller, K., Ooms, J., Robinson, D., Seidel,
D., Spinu, V., \ldots{} Yutani, H. (2019). Welcome to the tidyverse.
\emph{Journal of Open Source Software}, \emph{4}(43), 1686.
\url{https://doi.org/10.21105/joss.01686}

\end{CSLReferences}



\end{document}
